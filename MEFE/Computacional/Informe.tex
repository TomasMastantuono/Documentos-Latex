\documentclass[]{article}
\usepackage{amsmath}  % Permite eliminar enumeración en las ecuaciones
\usepackage[a4paper, margin=2cm]{geometry} % para los márgenes
\usepackage{graphicx}
\usepackage{indentfirst}
\usepackage{caption} %para poner captions en las figuras
\usepackage{float} %para que acepte floats y me permita poner [H] en las figuras

\title{Intervalos de confianza frecuentista y bayesianos para H0}
\author{Tomas Mastantuono, LU:522/23, DNI:45421427}
\date{}


\begin{document}
\maketitle

\section*{Objetivo}
En este trabajo se busca construir intervalos de confianza frecuentistas y bayesianos para el parámetro $H_0$ utilizando datos de cronómetros cósmicos. Los datos que me dan para
analizar son de la forma $\left( z, H\left(z\right), \sigma_{H\left(z\right)} \right)$, donde $z$ es el corrimiento al rojo, $H\left(z\right)$ es la tasa de expansión del universo
a ese corrimiento y $\sigma_{H\left(z\right)}$ es el error asociado a $H\left(z\right)$.

El modelo cosmológico utilizado es el $\Lambda CDM$, el cual me relaciona la tasa de expansión del universo con el parámetro $H_0$ y la densidad de la materia $\Omega_m$ a través
de la ecuación (\ref{fig: ajuste modelo H}).

\begin{equation}
    H\left(z\right) = H_0 \sqrt{ \Omega_m \left(1 + z\right)^3 + \left(1 - \Omega_m\right) }
    \label{eq: modelo teórico}
\end{equation}

\section{Datos observacionales $H\left(z\right)$ vs $z$}
Como primera actividad se pide un gráfico de los datos observacionales $H\left(z\right)$ en función de $z$, el cual se muestra en la Figura \ref{fig: ajuste modelo H}, junto con
un ajuste utilizando la expresión dada por la ecuación (\ref{eq: modelo teórico}).

\begin{figure}[H]
    \centering
    \includegraphics[width = 0.8\linewidth]{código/Imagenes/Problema_0.pdf}
    \captionsetup{justification=centering}
    \captionsetup{font=footnotesize}
    \caption{El primer gráfico corresponde a la tasa de expansión del universo $H\left(z\right)$ en función al corrimiento al rojo $z$ junto con su ajuste correspondiente utilizando la 
    ecuación (\ref{eq: modelo teórico}). El segundo gráfico corresponde a los residuos obtenido del ajuste y los datos, los cuales parecen no seguir ninguna tendencia específica.}
    \label{fig: ajuste modelo H}
\end{figure} 

Para realizar el ajuste que se muestra en la Figura utilicé $curvefit$, la cual es una función de una librería de $Scipy$. Del ajuste logré obtener parámetros óptimos para $H_0$ y para $\Omega_m$, los
cuales están dados por $H_0 = \left( 68 \pm 2\right)$ y $\Omega_m = \left( 0.31 \pm 0.04\right)$.

\section{Intervalo de confianza frecuentista}
A partir de este punto se pide explícitamente que se fije el valor de $\Omega_m$ a $0.3$ y que todo el análisis siguiente sea solo para $H_0$. Para construir un intervalo de
confianza frecuentista, tomé $M = 20$ valores equispaciados dentro del rango $\left[60, 80\right]$, esta decisión fue tomada pensando en el momento de tener que visualizar el
cinturón de confianza frecuentista, evitándome tener de forma inesperada valores acumulados de $H_0$ los cuales me agreguen sesgo a la interpretación final.

Para cada uno de estos valores tomados de $H_0$, realicé $N = 2000$ conjuntos de datos sintéticos construidos por la ecuación (\ref{eq: datos_sinteticos}), 
utilizando los mismos valores $z_i$ y $\sigma_{H\left(z_i\right)}$ de las mediciones. Tener en cuenta que $\epsilon_i \sim N\left( 0, \sigma_{H\left( z_i \right)} \right)$, por lo que
su único objetivo es agregar ruido gaussiano a los datos sintéticos.

\begin{equation}
    H^{sint}_i = H_0 \sqrt{\Omega_m \left( 1 + z_i \right)^3 + \left( 1 - \Omega_m \right)} + \epsilon_i
    \label{eq: datos_sinteticos}
\end{equation}

Por lo que hasta el momento tengo $M$ valores diferentes de $H_0$, de los cuales para cada uno de estos realicé $N$ sets de datos sintéticos. Todo esto tiene como objetivo lograr
que para cada valor de $H_0$ me pueda construir un cinturón de confianza del $68\%$ y del $95\%$ en función de un estimador $\hat{H}_0$. Por la forma en la que fueron construidos
los datos sintéticos, los cuales según la ecuación (\ref{eq: datos_sinteticos}) se puede notar un término del cual no depende de ninguna variable aleatoria y un término sumado que
posee errores gaussianos, utilizo cuadrados mínimos para estimar el $H_0$ de cada una de los $N$ sets y repito para los $M$ valores, es decir, uso $curvefit$ y ajusto con la ecuación 
(\ref{eq: modelo teórico}) los $N$ sets de datos sintéticos, obteniendo $N$ valores de $\hat{H}_0$ para cada uno de los $M$. De esta forma logro obtener numéricamente la distribución 
de $\hat{H}_0$ para cada uno de los $M$ del rango mencionado, por ejemplo se muestra en la Figura \ref{fig: histograma_h0} el caso para el primer $H_0$ dentro del rango 
$\left[60, 80\right]$, donde el primer gráfico muestra contenido un intervalo central del $68\%$ de confianza, mientras que el segundo un $95\%$ de confianza.

\begin{figure}[H]
    \centering
    \makebox[\textwidth][]{
        \includegraphics[scale=0.25]{código/Imagenes/Item_4.pdf}
    }
    \captionsetup{justification=centering}
    \captionsetup{font=footnotesize}
    \caption{Ambos gráficos corresponde a la misma simulación de datos sintéticos correspondientes de $H_0$. El gráfico de la izquierda muestra el intervalo que encierra el
    $68\%$ de probabilidad central. En cambio, el gráfico de la derecha encierra el $95\%$ de probabilidad centrada.}
    \label{fig: histograma_h0}
\end{figure}

Este proceso fue realizado para los $M = 20$ valores de $H_0$, donde para cada uno de estas distribuciones obtuve los estimadores $\hat{H}_{Min}$ y $\hat{H}_{Max}$ que marcan los límites
del intervalo del $68\%$ y $95\%$ de confianza central.

Utilizando los $M$ valores de $H_0$, y los estimadores límite de cada distribución puedo construir los cinturones de confianza frecuentista para el $68\%$ y el $95\%$ de confianza, tal
como se muestra en la Figura \ref{fig: cinturon_frecuentista}.

\begin{figure}[H]
    \centering
    \makebox[\textwidth][]{
        \includegraphics[scale = 0.7]{código/Imagenes/Intevalo_frecuentista.pdf}
    }
    \captionsetup{justification=centering}
    \captionsetup{font=footnotesize}
    \caption{Se muestra $H_0$ en función de $\hat{H}_0$. El cinturón de color verde corresponde al intervalo el cual encierra una cobertura del $95\%$@CL. 
    Por otro lado, el cinturón de color azul corresponde al intervalo de confianza que encierra un $68\%$@CL.}
    \label{fig: cinturon_frecuentista}
\end{figure}

Como me interesaba dar un intervalo de confianza frecuentista, tenía que fijarme donde interseca mi valor obtenido del ajuste con el cinturón correspondiente al $68\%$ o al $95\%$ de 
confianza. Para esto usé una función de $numpy$ llamada $interp$ donde al accederle el valor óptimo obtenido del ajuste de la Figura \ref{fig: ajuste modelo H} realizaba una 
interpolación con los valores propios del cinturón, logrando darme un punto aproximado que pertenezca al conjunto del intervalo. Tal como se muestra en la figura, 
obtuve $H_0 = 68_{-3.02}^{+3.19}$ $95\%$@CL, y $H_0 = 68_{-1.67}^{+1.63}$ $68\%$@CL.

Desde una interpretación frecuentista, el cinturón de $\alpha\%$ de confianza me dice que solo $\alpha$ de cada 100 veces voy a obtener a un intervalo que
contenga al valor real de $H_0$, por lo que no puedo concluir mas desde esta interpretación.

\section{Intervalo de credibilidad bayesiano}
La consigna me pide escribir la expresión de la $posterior$ para $H_0$, aún manteniendo fija $\Omega_m$, y suponiendo errores gaussianos e independientes. Por lo que si
utilizo el Teorema de Bayes, la expresión de la $posterior$ está dada por la ecuación (\ref{eq: T_bayes}).

\begin{equation}
    f\left(H_0 | \vec{x}\right) = \frac{\prod_{i}^{n} f\left( x_i | H_0 \right) \pi\left(H_0\right)}{\prod_{i}^{n} \int f\left( x_i | H_0 \right) \pi\left(H_0\right)d\left(H_0\right)} 
    \label{eq: T_bayes}
\end{equation}

Donde en esta ecuación $f\left(H_0 | \vec{x}\right)$ referencia a la función de distribución de $H_0$ dado que medí $\vec{x}$, mientras que la función $\pi\left(H_0\right)$
es el $prior$ que contiene la información hasta el momento de esa variable. Además a la integral que se encuentra en el denominador se la suele llamar $Z$, o también
evidencia. Hay que tener en cuenta que el enunciado me pide que asuma $f\left(x_i | H_0\right) \sim G\left( H\left(z_i\right), \sigma = \sigma_{H\left(z_i\right)} \right)$. 

Por lo que el primer problema numérico al cual me enfrento es hallar la función que se encuentra en el numerador de la ecuación (\ref{eq: T_bayes}). El primer paso que realicé fue
definirme una función la cual solo necesite los datos medidos y un único valor para $H_0$, para cada dato de las mediciones realizadas, y el $H_0$ colocado, me creo una distribución
gaussiana centrada en la ecuación (\ref{eq: modelo teórico}) evaluado en cada punto $z_i$ y con dispersión $\sigma_{H\left(z_i\right)}$. De todo esto obtengo la $pdf$ en el punto
$H_i$ que obtuve de la medición. Este proceso lo repito para todos los datos de las mediciones, y al ser todas independientes multiplico las funciones de distribución.

Para poder dar la función de distribución del parámetro $H_0$ necesitaba calcular $Z$, para realizar esa integral utilicé una función de integración de $Scipy$ llamada $quad$, la
cual al darle simplemente la función a integrar y el intervalo de integración devuelve el resultado final. Como es un integral numérica, pero los intervalos teóricos son en todos
los posibles valores para el parámetro, tomé un atajo y para cada prior pedido por el ejercicio visualicé previamente hasta que punto hay un aporte significativo, por lo que utilicé como
intervalo de integración solo esas regiones.  Finalmente, utilizando la ecuación (\ref{eq: T_bayes}) logré construir un $posterior$ utilizando en un caso el $prior$ $U\left[60, 80\right]$, 
y en otro un $prior$ $G\left(70, 5\right)$, tal como se muestra en la Figura \ref{fig: intervalo_bayes}.

\begin{figure}[H]
    \centering
    \makebox[\textwidth][]{
        \includegraphics[scale = 0.7]{código/Imagenes/Bayes_graficos.pdf}
    }
    \captionsetup{justification=centering}
    \captionsetup{font=footnotesize}
    \caption{En azul se muestra la función de distribución o $posterior$ para $H_0$ dadas las mediciones $\vec{x}$, la cual surge de suponer un $prior$ con distribución 
    $U\left[60, 80\right]$. Mientras que en naranja se puede ver la función de distribución para $H_0$ o $posterior$ dadas las mediciones $\vec{x}$, la cual surge de suponer
    un $prior$ con distribución $G\left[70, 5\right]$. Con línea punteada se muestra la $Esperanza$ de cada distribución respectivamente.}
    \label{fig: intervalo_bayes}
\end{figure}

De la $posterior$ proveniente de un $prior$ con distribución $U \left[60, 80\right]$, se obtiene un intervalo de $H_0 = \left(69.03 \pm 1.60\right)68\%$@CL y
$H_0 = \left(69.03 \pm 3.10\right)95\%$@CL. Mientras que la $posterior$ proveniente de un $prior$ con distribución $G\left[70, 5\right]$ se puede obtener un intevalo
de $H_0 = \left(69.12 \pm 1.60\right)68\%$@CL y $H_0 = \left(69.12 \pm 3.10\right)95\%$@CL. 

Como se púede ver, los resultados no son muy diferentes y poseen valores que se intersecan, por lo que si bien a simple vista pareciera que las distribuciones son notariamente diferentes, 
realmente los resultados comparten un gran intervalo entre ellos. El $prior$ uniforme en principio solo agrega información de que el valor de $H_0$ está contenido en el intervalo 
$\left[60, 80\right]$, mientras que el $prior$ gaussiano parece agregar mas información y que el verdadero valor de $H_0$ está contenido en una campana centrada en $70$ y con 
dispersión de $\sigma = 5$.

\section*{Conclusiones}
En conclusión obtuve en ambas interpretaciones intervalos los cuales se solapan en ciertos valores. Pude ver que la interpretación frecuentista solo me habla de que en ciertos casos voy a obtener
un intervalo que contenga al valor real del parámetro, y me da un intervalo para el estimador medido. Mientras que la interpretación bayesiana me agrega de forma explícita conocimiento 
previo sobre el parámetro que quiero conocer.

\end{document}