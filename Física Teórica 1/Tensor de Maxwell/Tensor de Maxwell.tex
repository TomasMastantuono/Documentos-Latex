\documentclass[]{article}
\usepackage{amsmath}  % Permite eliminar enumeración en las ecuaciones
\usepackage[a4paper, margin=2cm]{geometry} % para los márgenes

\title{\Huge Tensor de Maxwell}
\author{}
\date{}

\begin{document}
\maketitle

\section*{Tensor de Levi-Civita}
No tengo ni idea que es un tensor pero estas son algunas de las propiedades que voy a necesitar
para hacer la cuentas:

\begin{itemize}
    \item $\left[ \vec{a} \times \vec{b} \right]_i = \epsilon_{ijk} a_j b_k$
    \item $\epsilon_{ijk} = \epsilon_{jki}$ solo con permutaciones cíclicas
    \item $\epsilon_{jki} \epsilon_{jmn} = \delta_{km} \delta_{in} - \delta_{kn} \delta_{im}$
\end{itemize}

\section*{Ecuaciones de Maxwell en vacio}
Las ecuaciones de Maxwell en vacio están dadas por:

\begin{table}[h]
\begin{tabular}{cc}

% Divergencia de E
\begin{minipage}{0.4\textwidth}
\begin{equation}
    \vec{\nabla} \cdot \vec{E} = 4 \pi \rho
    \label{eq: div E}
\end{equation}
\end{minipage}

&
\quad

% Rotor de E
\begin{minipage}{0.4\textwidth}
\begin{equation}
    \vec{\nabla} \times \vec{E} = -\frac{1}{c} \frac{\partial \vec{B}}{\partial t}
    \label{eq: rot E}
\end{equation}
\end{minipage}


\\

% div de B
\begin{minipage}{0.4\textwidth}
\begin{equation}
    \vec{\nabla} \cdot \vec{B} = 0
    \label{eq: div B}
\end{equation}
\end{minipage}

&
\quad

% rot de B
\begin{minipage}{0.4\textwidth}
\begin{equation}
    \vec{\nabla} \times \vec{B} = \frac{4 \pi}{c} \vec{J} + \frac{1}{c} \frac{\partial E}{\partial t}
    \label{eq: rot B}
\end{equation}
\end{minipage}

\end{tabular}
\end{table}

\section*{Fuerza de Lorentz}
La fuerza de Lorentz en unidades Gaussianas está dado por (\ref{eq: F_em sobre 1 carga}):

\begin{equation}
    \vec{F}_{EM} = q \left(\vec{E} + \frac{\vec{v}}{c} \times \vec{B}\right)
    \label{eq: F_em sobre 1 carga}
\end{equation}

Para el caso de la fuerza aplicada sobre una distribución de carga, en este 
caso volumétrica, tenemos que reemplazar (\ref{eq: dist vol}) y (\ref{eq: densidad de corriente volumetrica})
en (\ref{eq: F_em sobre 1 carga}) obteniendo (\ref{eq: F_em sobre distribuciones}):

\begin{equation}
    q = \rho dv
    \label{eq: dist vol}
\end{equation}

\begin{equation}
    \vec{J} = \rho \vec{v}
    \label{eq: densidad de corriente volumetrica}
\end{equation}

\begin{equation}
    \vec{F}_{EM} = \int_{v}^{} [\rho \vec{E} + \frac{\vec{J}}{c} \times \vec{B}] dv
    \label{eq: F_em sobre distribuciones}
\end{equation}


\section*{Momento lineal Electromagnético}

Usando la leyes de Newton, podemos escribir las ecuaciones de movimiento a partir de las fuerzas
que sienten estas. En el caso de una cierta distribución de cargas está dado por (\ref{eq: Newton completa}),
donde se tiene en cuenta la fuerza dada por (\ref{eq: F_em sobre distribuciones}) y una cierta
$\vec{F_0}$ que tiene en si todas las demás fuerzas externas:

\begin{equation}
    \frac{d\vec{P_{M}}}{dt} = \int_{v}^{} [\rho \vec{E} + \frac{\vec{J}}{c} \times \vec{B}] dv + \vec{F_0}
    \label{eq: Newton completa}
\end{equation}

En todo el resto de cosas no voy a tener en cuenta la fuerza $\vec{F_0}$, solo es un término que para este
tema molesta.

\begin{equation}
    \frac{d\vec{P_{M}}}{dt} = \int_{v}^{} [\rho \vec{E} + \frac{\vec{J}}{c} \times \vec{B}] dv
    \label{eq: Newton}
\end{equation}

Se puede reescribir el argumento del término correspondiente a la fuerza electromagnética utilizando
las ecuaciones (\ref{eq: rho usando maxwell}) y (\ref{eq: j usando maxwell}), logrando obtener 
(\ref{eq: corchete 1}):

\begin{table}[h]
\begin{tabular}{cc}

%Ecuacion del rho
\begin{minipage}{0.4\textwidth}
\begin{equation}
    \rho = \frac{1}{4 \pi} (\vec{\nabla} \cdot \vec{E})
    \label{eq: rho usando maxwell}
\end{equation}
\end{minipage}

\quad

%Ecuacion del J
\begin{minipage}{0.4\textwidth}
\begin{equation}
    \vec{J} = \frac{c}{4 \pi} (\vec{\nabla} \times \vec{B}) - \frac{1}{4 \pi} \frac{\partial \vec{E}}{\partial t}
    \label{eq: j usando maxwell}
\end{equation}
\end{minipage}

\end{tabular}
\end{table}

\begin{equation}
    \rho \vec{E} + \frac{\vec{J}}{c} \times \vec{B} = \frac{1}{4 \pi} (\vec{\nabla} \cdot \vec{E}) \vec{E} + 
    \left[ 
        \frac{1}{4 \pi c} (\vec{\nabla} \times \vec{B}) - \frac{1}{4 \pi} \frac{\partial \vec{E}}{\partial t}
    \right] \times \vec{B}
    \label{eq: corchete 1}
\end{equation}

Si utilizamos el siguiente "truco" (\ref{eq: truco 1}) podemos reescribir (\ref{eq: corchete 1}) como:


\begin{equation}
    \frac{\partial}{\partial t} \left[
        \vec{E} \times \vec{B}
    \right] =
    \partial_t \vec{E} \times \vec{B} + \vec{E} \times \partial_t \vec{B}
    \label{eq: truco 1}
\end{equation}

\begin{equation}
    \rho \vec{E} + \frac{\vec{J}}{c} \times \vec{B} =
    \frac{1}{4 \pi} (\vec{\nabla} \cdot \vec{E})\vec{E} + \frac{1}{c} \left[
        \frac{1}{4 \pi} (\vec{\nabla} \times \vec{B}) \times \vec{B} -
        \frac{1}{4 \pi c} (\partial_t \left[\vec{E} \times \vec{B}\right] - \vec{E} \times \partial_t \vec{B})
    \right]
    \label{eq: corchete 2}
\end{equation}

Si usamos (\ref{eq: rot E}) podemos reemplazar $\partial_t \vec{B}$ a $-c \vec{\nabla} \times \vec{E}$, 
de forma que reorganizando podemos obtener (\ref{eq: corchete 3}):

\begin{equation}
    \rho \vec{E} + \frac{\vec{J}}{c} \times \vec{B} =
    \frac{1}{4 \pi} (\vec{\nabla} \cdot \vec{E})\vec{E} + \frac{1}{c} \left[
        \frac{1}{4 \pi} (\vec{\nabla} \times \vec{B}) \times \vec{B} -
        \frac{1}{4 \pi c} (\partial_t \left[\vec{E} \times \vec{B}\right] - 
        \vec{E} \times c \vec{\nabla} \times \vec{E})
    \right]
    \label{eq: corchete 3}
\end{equation}

La cual podemos reorganizar según (\ref{eq: corchete 4}):

\begin{equation}
    \frac{1}{4 \pi} (\vec{\nabla} \cdot \vec{E})\vec{E} + 
    \frac{1}{4 \pi} (\vec{\nabla} \times \vec{E}) \times \vec{E} +
    \frac{1}{4 \pi} (\vec{\nabla} \times \vec{B}) \times \vec{B} -
    \frac{1}{4 \pi} \partial_t \left[ \vec{E} \times \vec{B} \right]
    \label{eq: corchete 4}
\end{equation}

Donde a (\ref{eq: corchete 4}) podemos sumar $0$, o lo que es lo mismo, 
$\frac{1}{4 \pi} (\vec{\nabla} \cdot \vec{B})\vec{B}$. De esta forma si colocamos esta expresión en
(\ref{eq: Newton}) podríamos separar la integral en los siguientes dos términos:

\begin{equation}
    \frac{d\vec{P_{M}}}{dt} + \frac{1}{4 \pi c} \int_{v}^{} \partial_t \left[
    \vec{E} \times \vec{B}
    \right] dv = 
    \int_{v}^{} \frac{1}{4 \pi} \left[
    (\vec{\nabla} \cdot \vec{E})\vec{E} + 
    (\vec{\nabla} \times \vec{E}) \times \vec{E} +
    (\vec{\nabla} \cdot \vec{B})\vec{B} +
    (\vec{\nabla} \times \vec{B}) \times \vec{B}
    \right] dv 
    \label{eq: Newton 2}
\end{equation}

Donde de (\ref{eq: Newton 2}) podemos definir el momento electromagnético como:

\begin{equation}
    \vec{P}_{EM} =
    \frac{1}{4 \pi c} \int_{v}^{}\left[
    \vec{E} \times \vec{B}
    \right] dv 
\end{equation}

Utilizando esto podemos reescribir (\ref{eq: Newton 2}) como:

\begin{equation}
    \partial_t (\vec{P}_M + \vec{P}_{EM}) =
    \int_{-\infty}^{+\infty} \frac{1}{4 \pi} \left[
    (\vec{\nabla} \cdot \vec{E})\vec{E} + 
    (\vec{\nabla} \times \vec{E}) \times \vec{E} +
    (\vec{\nabla} \cdot \vec{B})\vec{B} +
    (\vec{\nabla} \times \vec{B}) \times \vec{B}
    \right] dv
    \label{eq: Newton 3}
\end{equation}


Donde de (\ref{eq: Newton 3}) definimos la matriz de Maxwell a (\ref{eq: matriz de maxwell}):

\begin{equation}
    \vec{\nabla} \cdot \vec{T} =
    \left[
    (\vec{\nabla} \cdot \vec{E})\vec{E} + 
    (\vec{\nabla} \times \vec{E}) \times \vec{E} +
    (\vec{\nabla} \cdot \vec{B})\vec{B} +
    (\vec{\nabla} \times \vec{B}) \times \vec{B}
    \right]
    \label{eq: matriz de maxwell}
\end{equation}

Veamos en notación de índices lo que dice cada término del corchete, primeramente con el término 
correspondiente al rotor de $\vec{E}$:

\begin{equation}
    \left[
    (\vec{\nabla} \times \vec{E}) \times \vec{E}
    \right]_i =
    \epsilon_{ijk} (\epsilon_{jmn} \partial_m E_n)E_k = 
    \epsilon_{ijk} \epsilon_{jmn} E_k \partial_m E_n
    \label{eq: cosas de tensores 1}
\end{equation}

Donde en (\ref{eq: cosas de tensores 1}) la propiedad de permutación cíclica sobre tensor de Levi-Civita 
$\epsilon_{ijk}$ para llevarlo al tensor $\epsilon_{jki}$ poder aplicar la propiedad de las deltas:

\begin{equation*}
    \left[
    (\vec{\nabla} \times \vec{E}) \times \vec{E}
    \right]_i =
    \epsilon_{ijk} \epsilon_{jmn} E_k \partial_m E_n =
    \epsilon_{jki} \epsilon_{jmn} E_k \partial_m E_n =
    (\delta_{km} \delta_{in} - \delta_{kn} \delta_{im}) E_k \partial_m E_n
    \label{eq: cosas de tensores 2}
\end{equation*}

\begin{equation}
    \Leftrightarrow
    \left[
    (\vec{\nabla} \times \vec{E}) \times \vec{E}
    \right]_i =
    E_k \partial_k E_i - E_k \partial_i E_k
    \label{eq: indice rotor de E}
\end{equation}

Ahora veamos el término correspondiente a la divergencia de $\vec{E}$:

\begin{equation}
    \left[ (\vec{\nabla} \cdot \vec{E}) \vec{E} \right]_i =
    (\partial_j E_j)E_i
    \label{eq: indice divergencia de E}
\end{equation}

Antes de seguir con los términos correspondientes a $\vec{B}$ notemos que en (\ref{eq: matriz de maxwell})
posee esencialmente dos términos, uno vinculado con la suma de cosas de divergencias y rotores de $\vec{E}$,
y otro exactamente igual pero reemplazando $\vec{E}$ por $\vec{B}$. Por lo que si calculamos la suma con $\vec{E}$
de forma directa podemos obtener lo que ocurre con $\vec{B}$.

Para la siguiente cuenta notar que $ijk$ son índices mudos, por lo que puedo llamar $j \equiv k$ sin problemas:

\begin{equation*}
    \left[
    (\vec{\nabla} \cdot \vec{E}) \vec{E} + (\vec{\nabla} \times \vec{E}) \vec{E}
    \right]_i =
    \partial_j E_j E_i + E_k \partial_k E_i - E_k \partial_i E_k
\end{equation*}

\begin{equation*}
    \Leftrightarrow
    \left[
    (\vec{\nabla} \cdot \vec{E}) \vec{E} + (\vec{\nabla} \times \vec{E}) \vec{E}
    \right]_i =
    E_j \partial_j E_i + \partial_j E_j E_i - E_j \partial_i E_j
\end{equation*}

\begin{equation*}
    \Leftrightarrow
    \left[
    (\vec{\nabla} \cdot \vec{E}) \vec{E} + (\vec{\nabla} \times \vec{E}) \vec{E}
    \right]_i =
    \partial_j (E_i E_j) - E_j \partial_i E_j
\end{equation*}

\begin{equation*}
    \Leftrightarrow
    \left[
    (\vec{\nabla} \cdot \vec{E}) \vec{E} + (\vec{\nabla} \times \vec{E}) \vec{E}
    \right]_i =
    \partial_j (E_i E_j) - \frac{1}{2} \partial_i (E_j E_j)
\end{equation*}

\begin{equation}
    \Leftrightarrow
    \left[
    (\vec{\nabla} \cdot \vec{E}) \vec{E} + (\vec{\nabla} \times \vec{E}) \vec{E}
    \right]_i =
    \partial_i \left[
    (E_i E_j) - \frac{1}{2} (E_j E_j)\delta_{ij}
    \right]
    \label{eq: suma de div y rot de E}
\end{equation}

Si repetimos todos los mismos procesos con $\vec{B}$ hasta llegar a (\ref{eq: suma de div y rot de E}),
mirando (\ref{eq: matriz de maxwell}) y (\ref{eq: suma de div y rot de E}) podemos notar que nos queda de 
factor común $\partial$ en notación de índices, por lo que tomando el factor $(4 \pi)^{-1}$ de 
(\ref{eq: Newton 3}) podemos definir el Tensor de Maxwell como:

\begin{equation}
    T_{ij} = \frac{1}{4 \pi} \left[
    E_i E_j + B_i B_j - \frac{1}{2} \delta_{ij}(\vec{E}\vec{E} + \vec{B}\vec{B})
    \right]
    \label{eq: Tensor de maxwell}
\end{equation}

Por lo que volviendo a (\ref{eq: Newton 3}) podemos reescribir en notación de índices como:

\begin{equation*}
    \left[
    \partial_t (\vec{P}_M + \vec{P}_{EM})
    \right]_i = 
    \int_{v}^{} (\partial_j T_{ij}) dv
\end{equation*}

Donde si usamos Stokes (o teorema de la divergencia) obtenemos:

\begin{equation*}
    \left[
    \partial_t (\vec{P}_M + \vec{P}_{EM})
    \right]_i = 
    \int_{s(v)}^{} T_{ij} n_j ds
\end{equation*}

\begin{equation}
    \Leftrightarrow
    \frac{\partial}{\partial t} (\vec{P}_M + \vec{P}_{EM}) = 
    \int_{s(v)}^{} \vec{\vec{T}} \hat{n} ds
    \label{eq: Newton Final}
\end{equation}

Se puede calcular como es la proyección sobre alguna dirección del
tensor de Maxwell utilizando la ecuación (\ref{eq: Tensor de maxwell}), esto es
muy util para resolver la integral dada en ecuación (\ref{eq: Newton Final}) de 
cada problema en específico:

\begin{equation}
    \vec{\vec{T}} \cdot \hat{n} =
    \frac{1}{4 \pi} \left[
        \vec{E} \left(\vec{E} \cdot \hat{n}\right) +
        \vec{B} \left(\vec{B} \cdot \hat{n}\right) -
        \frac{1}{2} \left(E^2 + B^2\right) \hat{n}
    \right]
\end{equation}

\end{document}